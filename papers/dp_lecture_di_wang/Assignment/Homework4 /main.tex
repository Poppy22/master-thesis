\documentclass[11pt]{article}
\usepackage{amsmath}
\usepackage{datetime}
\usepackage{amssymb}
\usepackage{hyperref}

\usepackage{algorithm}% http://ctan.org/pkg/algorithms
\usepackage{algpseudocode}% http://ctan.org/pkg/algorithmicx
\newcommand{\numpy}{{\tt numpy}}    % tt font for numpy
\newdate{date}{26}{12}{2022}
\newtheorem{theorem}{Theorem}
\newtheorem{lemma}[theorem]{Lemma}
\newtheorem{proposition}[theorem]{Proposition}
\newtheorem{claim}[theorem]{Claim}
\newtheorem{corollary}[theorem]{Corollary}
\newtheorem{definition}[theorem]{Definition}
\date{\displaydate{date}}
\topmargin -.5in
\textheight 9in
\oddsidemargin -.25in
\evensidemargin -.25in
\textwidth 7in

\begin{document}

% ========== Edit your name here
\author{Deadline: 11th December, 2022}
\title{Homework 4}
\maketitle

\medskip
% ========== Begin answering questions here
\begin{enumerate}


\item (3pts) Proof Theorem 10.8 in Lecture 10. You can check reference [2] in the lecture. 
\item (3.5 pts) Try to use the Opacus library to implement the DP-SGD algorithm for some deep learning tasks in PyTorch. You can find a tutorial at \url{https://opacus.ai/}. Write a report for your experimental setting and results. 
\item (3.5 pts) Try to use the Tensorflow Privacy to implement the DP-SGD algorithm for some deep learning tasks in Tensorflow. You can find a tutorial at \url{https://www.tensorflow.org/responsible_ai/privacy/guide}. Write a report for your experimental setting and results. 

\end{enumerate}

\end{document}
\grid
\grid