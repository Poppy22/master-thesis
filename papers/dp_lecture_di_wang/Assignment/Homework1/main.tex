\documentclass[11pt]{article}
\usepackage{amsmath}
\usepackage{datetime}
\usepackage{amssymb}
\newcommand{\numpy}{{\tt numpy}}    % tt font for numpy
\newdate{date}{15}{09}{2022}
\newtheorem{theorem}{Theorem}
\newtheorem{lemma}[theorem]{Lemma}
\newtheorem{proposition}[theorem]{Proposition}
\newtheorem{claim}[theorem]{Claim}
\newtheorem{corollary}[theorem]{Corollary}
\newtheorem{definition}[theorem]{Definition}
\date{\displaydate{date}}
\topmargin -.5in
\textheight 9in
\oddsidemargin -.25in
\evensidemargin -.25in
\textwidth 7in

\begin{document}

% ========== Edit your name here
\author{Deadline: 29th September, 2022}
\title{Homework 1}
\maketitle

\medskip

% ========== Begin answering questions here
\begin{enumerate}
\item (Properties of Laplace Distribution) 
Prove that if $Z\sim \text{Lap}(\lambda)$ us a Laplace-distributed random variable, we have 
\begin{itemize}
	\item $\sqrt{\mathbb{E}(Z^2)}=\sqrt{2}\lambda $
    \item For every $t>0$: $\mathbb{P}(z> \lambda t)\leq \exp(-t)$. 
\end{itemize}
\item (Global Sensitivity) ) For all of the following cases, assume we have a dataset $D=\{x_1, \cdots, x_n)\in \mathcal{X}^n$ and a function $f: \mathcal{X}\mapsto \mathbb{R}^d$. For each of the following function $f$ and data domains $\mathcal{S}$,  give as tight a bound as you can one the global sensitivity of the function $f$. If the sensitivity is not bounded, answer $\infty$. 
\begin{enumerate}
\item The high dimensional mean $f(D)=\frac{1}{n}\sum_{i=1}^n x_i$ where $\mathcal{X}=\{v\in \mathbb{R}^d: \|v\|_1\leq 1\}$. 
\item The unnormalized covariance matrix when $\mathcal{X}=\{v\in \mathbb{R}^d: \|v\|_1\leq 1\}$. Here $f(D)=\sum_{i=1}^n x_i x_i^T$ is a $d\times d$ symmetric matrix. To measure the sensitivity, we think $f(D)$ as a single vector of length $d^2$. 
\item The median $f(D)=\text{median}(x_1, \cdots, x_n)$ when $\mathcal{X}=[0, 1]$. 
\item Suppose we have a fixed set of vertices $V$ (independent of the dataset). Our dataset is a list of edges: each $x_i$ is a pair of vertices $(u, v)$ (so that $\mathcal{X}=V\times V$. Let $G_D$ be the resulting graph, and let $f(D)$ be the number of connected components in $G_D$ (A connected component or simply component of an undirected graph is a subgraph in which each pair of nodes is connected with each other via a path).  

\end{enumerate}
\item  (Gumbel Max Trick) Show that Report Noisy Max algorithm with parameter $\beta=\frac{2\Delta}{\epsilon}$ generates 
exactly the same distribution as the exponential mechanism.
\item (Random Response and Laplacian Mechanism) This is an experimental question. In the class we showed the random response and Laplacian mechanism for answer the query $f(D)=\frac{1}{n}\sum_{i=1}^n x_i$ for each $x_i\in \{0, 1 \}$. Try to implement these two mechanisms and analyze their utilities with different sample size $n$ and $\epsilon$. You can design the data generation process by your self. Write the brief report on your findings. 

\end{enumerate}


\end{document}
\grid
\grid