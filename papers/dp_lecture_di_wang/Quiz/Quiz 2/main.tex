\documentclass[11pt]{article}
\usepackage{amsmath}
\usepackage{amssymb}
\newcommand{\numpy}{{\tt numpy}}    % tt font for numpy
\usepackage{amsmath,amsfonts,amssymb,graphicx,mathtools,flexisym}
\usepackage{algorithm}% http://ctan.org/pkg/algorithms
\usepackage{algpseudocode}% http://ctan.org/pkg/algorithmicx
\topmargin -.5in
\textheight 9in
\oddsidemargin -.25in
\evensidemargin -.25in
\textwidth 7in

\begin{document}

% ========== Edit your name here
\author{Name and KAUST ID}
\title{Quiz 2}
\maketitle

\medskip

% ========== Begin answering questions here
\begin{enumerate}

\item Consider the following statements. Judge whether each of them is true or false. You don’t need to explain the reason.
\begin{itemize}

\item For $(\epsilon, \delta)$-DP algorithms, the privacy budget $\delta$ must satisfy $\delta\ll \frac{1}{n}$. Equivalently, there is an $(\epsilon=0, \delta)$-DP algorithm such that if $\delta\gg \frac{1}{n}$ then with high probability its output will reveal individuals information. 
\item Consider the following the statement:

For a mechanism $A: \mathcal{X}^n \mapsto \mathcal{Y}$, a pair of neighboring data $D\sim D'$, defines the sets 
\begin{equation}
    \text{Good}=\{y\in \mathcal{Y}: \frac{\mathbb{P}(A(D)=y) }{\mathbb{P}(A(D')=y }\leq e^\epsilon\}, \text{Bad}= \mathcal{Y}-  \text{Good}. 
\end{equation}
Then $A$ is $(\epsilon, \delta)$-DP if and only if $\mathbb{P}(A(D)\in  \text{Bad})\leq \delta$ for every pair of neighboring datasets. Note that is $A(D)$ and $A(D')$ are continuous distributions then we just replace the probability to the probability density functions. 
\item For a given privacy budget $\epsilon$, the error of Laplacian mechanism to achieve $\epsilon$-DP is always smaller than the error of Gaussian mechanism to achieve $(\epsilon, \delta=\frac{1}{n})$-DP.
\item We know that Differential privacy has the subsampling property. However, different subsampling approaches may lead different level of privacy guarantees. 
\item Like the approximate (or $(\epsilon, \delta)$) DP, R\'{e}nyi-DP also has a similar form the advanced composition theorem. 
\end{itemize}

\end{enumerate}
 
% ========== Continue adding items as needed



\end{document}
\grid
\grid