\documentclass[11pt]{article}
\usepackage{amsmath}
\usepackage{amssymb}
\newcommand{\numpy}{{\tt numpy}}    % tt font for numpy
\usepackage{amsmath,amsfonts,amssymb,graphicx,mathtools,flexisym}
\usepackage{algorithm}% http://ctan.org/pkg/algorithms
\usepackage{algpseudocode}% http://ctan.org/pkg/algorithmicx
\topmargin -.5in
\textheight 9in
\oddsidemargin -.25in
\evensidemargin -.25in
\textwidth 7in

\begin{document}

% ========== Edit your name here
\author{Name and KAUST ID}
\title{Quiz 3}
\maketitle

\medskip

% ========== Begin answering questions here
\begin{enumerate}

\item Consider the following statements. Judge whether each of them is true or false. You don’t need to explain the reason.
\begin{itemize}

\item Compared with the exponential mechanism or the noisy-max mechanism, one of the advantages of the Sparse Vector technique is it can deal with the case where the queries are online. 

\item In the parse Vector technique, besides the query values, we also need to add noise to threshold to make the algorithm satisfies DP. 

\item  To address the case where the query has large or unbounded sensitivity, in Lecture 10 we introduced two general approaches: PTR method and the smooth sensitivity. However, in general these two approaches are inefficient (that is, it may be NP-hard). But for some specific queries, these two approaches are efficient (that is there is some algorithm with polynomial time complexity). 
\item For one-dimensional average estimation where each $x_i\in \{0, 1\}$, its estimation error in the $\epsilon$-LDP model is always greater than it in the $\epsilon$-DP model. 
\item In the definition of shuffle DP, we only want to ensure the whole protocol to satisfy DP instead of LDP. 
\end{itemize}

\end{enumerate}
 
% ========== Continue adding items as needed



\end{document}
\grid
\grid